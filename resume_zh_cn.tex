\documentclass[11pt]{article}

% --- Modify margins --- %
\usepackage{geometry}
\geometry{a4paper,scale=0.8}

% --- Set English Font --- %
% \renewcommand{\familydefault}{\sfdefault}
% --- Set Chinese Font --- %
\usepackage{xeCJK}
% \setCJKfamilyfont{yh}{Microsoft YaHei}                    %微软雅黑 yh
% \newcommand{\yh}{\CJKfamily{yh}}

% --- Involved packages --- %
\usepackage[T1]{fontenc}
\usepackage{linespacing_fix}
\usepackage{url}

% disable indent globally
\setlength{\parindent}{0pt}
% some general improvements, defines the XeTeX logo
\RequirePackage{xltxtra}
% use xifthen
\RequirePackage{xifthen}

\RequirePackage{titlesec}
\RequirePackage{enumitem}
\setlist{noitemsep} % removes spacing from items but leaves space around the whole list
%\setlist{nosep} % removes all vertical spacing within and around the list
\setlist[itemize]{topsep=0.25em, leftmargin=1.5pc}
\setlist[enumerate]{topsep=0.25em, leftmargin=1.5pc}
\RequirePackage[super]{nth}

\titleformat{\section}              % Customise the \section command 
    {\Large\scshape\raggedright}    % Make the \section headers large (\Large),
                                    % small capitals (\scshape) and left aligned (\raggedright)
    {}{0em}                      % Can be used to give a prefix to all sections,    like 'Section ...'
    {}                           % Can be used to insert code before the heading
    [\titlerule]                 % Inserts a horizontal line after the heading
\titlespacing*{\section}{0cm}{*1}{*1}

\titleformat{\subsection}
    {\large\raggedright}
    {}{0em}
    {}
\titlespacing*{\subsection}{0cm}{*1}{*0.5}

\newcommand{\datedsection}[2]{%
    \section[#1]{#1 \hfill #2}%
}
\newcommand{\datedsubsection}[2]{%
    \subsection[#1]{#1 \hfill #2}%
}
\newcommand{\datedline}[2]{%
    {\par #1 \hfill #2 \par}%
}

\newcommand{\name}[1]{
    \centerline{\Huge\scshape\textbf{#1}}
    \vspace{1.2ex}
}
\newcommand{\contactInfo}[3]{
    \centerline{
        \sffamily\large{
            \ 手机:{#1} \textperiodcentered
            \ 邮箱:{#2} \textperiodcentered
            \ {Github}:\normalsize\url{#3}
            }
        } 
    \vspace{1.2ex}
}
\newcommand{\otherInfo}[4]{
    \centerline{\sffamily\large{\ {#1}}
    \ifthenelse{\isempty{#2}}%
        { } % if {#2} is empty
    {\textperiodcentered\ \ {#2} } % homepage, no space before \text...
    \ifthenelse{\isempty{#3}}%
        { } % if {#3} is empty
        {\textperiodcentered\ \ {#3} } % homepage, no space before \text...
    \ifthenelse{\isempty{#4}}%
        { } % if {#3} is empty
        {\textperiodcentered\ \ {#4} }
    }
    \vspace{0.7ex}
}
\newcommand{\role}[2]{
    {\par \textit{#1} ~ #2 \par}
    \vspace{0.5ex}
}

\begin{document}

% personal information
\name{张子杰}
\contactInfo{+86 188-5182-2129}{z.zijie@outlook.com}{https://github.com/z-zijie}
% \contactInfo{+1 608-949-4685}{zijie.us@gmail.com}

\section{\textbf{教育背景}}
\datedsubsection{
    \textbf{威斯康星大学麦迪逊分校},
    \textit{应用数学-硕士}}
    {2019.09 - 2021.05}
    \begin{itemize} [parsep=1ex]
        \item \textbf{University of Wisconsin-Madison}
        \item GPA: 3.88/4.0
    \end{itemize}

\datedsubsection{
    \textbf{南京大学},
    \textit{统计学-学士(数学系)}}
    {2016.09 - 2020.05}
    \begin{itemize} [parsep=1ex]
        \item GPA: 3.95/5.0, 前30\%.
    \end{itemize}

\hspace*{\fill}
\section{\textbf{项目经历}}
\datedsubsection{
    \textbf{口罩识别},
    \textit{机器学习}
    (sklearn, pytorch)}
    {2020.10 - 2020.12}
    \begin{itemize} [parsep=1ex]
        \item 使用\textbf{opencv}基于\textit{DNN}的人脸检测,对有口罩遮挡的人脸进行准确追踪。
        \item 使用自建数据集,训练\textit{MaskDetector}分类器。
        \item 个人项目
    \end{itemize}

\datedsubsection{
    \textbf{泰坦尼克号幸存预测},
    \textit{机器学习}}
    {2020.09 - 2020.11}
    \begin{itemize} [parsep=1ex]
        \item \textbf{Kaggle Titanic}
        \item 完全从零开始手写\textit{Linear Regression, Logistic Regression, Decision Trees, Nearest Neighbors, Naive Bayes, Support Vector Machines}等基础机器学习算法。
        \item 比较各个算法优劣,优化\textit{Nearest Neighbors}搜索,对\textit{Decision Trees}进行预剪枝、后剪枝。
        \item 个人项目,University of Wisconsin-Madison, COMP SCI 760, \textit{Prof.Daniel Pimentel-Alarcón}.
    \end{itemize}

\datedsubsection{
    \textbf{Fortran MPI多线程计算}
    }
    {2019.09 - 2019.12}
    \begin{itemize} [parsep=1ex]
        \item 对流方程, 热传导方程等偏微分方程
        \item \textbf{Heat equation, advection equation}
        \item 计算数学,有限元
        \item 主要使用的方法:\textit{Beam–Warming Method, Lax–Wendroff Method, Crank–Nicolson Method, Adams–Moulton Method}
        \item 主要工具: \textbf{MATLAB, FORTRAN, OpenMPI, Python}.
    \end{itemize}


\hspace*{\fill}
\section{\textbf{科研经历}}
\datedsubsection{
    \textbf{南京大学金融创新实验室},
    \textit{区块链算法设计}
    }
    {2018.09 - 2018.12}
    \begin{itemize} [parsep=1ex]
        \item 为高速交易设计了一个共轭routing和账户平衡算法.
        \item 在\textit{Tianhe}超算上用MATLAB进行仿真和数值优化,使交易成功率超过90\%.
        \item 为了提高计算效率,使用\textit{MEX C},在MATLAB中使用C/C++加速.
        \item 该项目的论文(三作),\textbf{Photon State-Channel Architecture with AI Routing Optimization}.
    \end{itemize}


% \hspace*{\fill}
\section{\textbf{工作经历}}
\datedsubsection{
    \textbf{口译},
    \textit{会议翻译}
    }
    {2020.11}
    \begin{itemize} [parsep=1ex]
        \item 为\textbf{南京大学}代表团访问\textbf{威斯康星大学麦迪逊分校}提供口译服务。
        \item 为与Exact Sciences生物医药癌症检测技术交流会参会者提供实时口译。
        \item 陪同参观位于威斯康星州麦迪逊市的研究所Discovery Building,并提供即时翻译交流。
    \end{itemize}


% \hspace*{\fill}
\section{\textbf{技能}}
\begin{itemize} [parsep=1ex]
    \item 编程语言: C/C++, Python, MATLAB, SQL, R, Fortran.
    \item 排版系统: \LaTeX.
    \item 平台: Linux, Windows.
    \item 语言能力: 中文(母语), 英文(流利)。
\end{itemize}

\section{\textbf{获奖情况}}
\begin{itemize} [parsep=1ex]
    \item Honorable Mention in ICM(美国大学生数学建模竞赛H奖)。
    \item 2017全国大学生数学竞赛(数学组),二等奖。
\end{itemize}


\end{document}